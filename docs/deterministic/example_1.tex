\documentclass[12pt]{article}

% a font designed for accessibility
% \usepackage[default]{sourcesanspro} 
\usepackage[default]{lato}
% \usepackage[default]{cabin}
% \usepackage{newtxsf} % if you want an aesthetic match on maths

\usepackage[T1]{fontenc}

\usepackage{amsmath,amsthm,amssymb,setspace,enumitem,epsfig,titlesec,verbatim,color,array,eurosym,multirow,blkarray,lscape,soul,graphicx,cite,comment,nicefrac,lineno,epstopdf,accents}
\usepackage[bf,small]{caption}
\usepackage[top=2cm,left=2cm,right=2cm,bottom=2cm]{geometry}
\usepackage[super,sort&compress,comma]{natbib}

\newtheorem{theorem}{Theorem}
\newtheorem{claim}[theorem]{Claim}
\newtheorem{corollary}{Corollary}
\newtheorem{definition}{Definition}
\newtheorem{example}[theorem]{Example}
\newtheorem{lemma}{Lemma}
\newtheorem{proposition}{Proposition}
\newtheorem{assumption}{Assumption}

% --- nadiah added
\usepackage{parskip} % I just find it easier to read
\usepackage{mathtools} % allows \coloneq
\usepackage{annotate-equations} % so I can point to reasoning

% tables
\usepackage{tabulary} % auto word wrap
\usepackage{booktabs} % toprule, midrule, bottomrule

\usepackage{hyperref}
\hypersetup{
    colorlinks   = true, %Colours links instead of ugly boxes
    urlcolor     = blue, %Colour for external hyperlinks
    linkcolor    = blue, %Colour of internal links
    citecolor   = blue %Colour of citations
}

% nicer courier links
\usepackage{microtype}
\usepackage{xurl}
\usepackage{inconsolata}

% more break points:
\newcommand{\ttpath}[1]{%
  \begingroup
  \catcode`\_=12
  \path{#1}%
  \endgroup
}

\usepackage{xcolor}
\definecolor{maroon}{RGB}{153,0,51}
\definecolor{darkgreen}{RGB}{0,100,0}
\newcommand{\m}[0]{\color{maroon}}
\newcommand{\g}[0]{\color{darkgreen}}

\usepackage{caption}
\captionsetup{font=footnotesize} % smaller captions

% these two packages are needed for the shaded results
\usepackage{amsthm}
\usepackage{framed}

% nicer code snippets
% ---
\usepackage{listings}
\lstset{
    % language=Python,
    showspaces=false,
    basicstyle=\small\tt,
    keywordstyle=\color{blue},
    commentstyle=\color{darkgreen},
    stringstyle=\color{darkgreen},
    breaklines=true,
    backgroundcolor=\color{yellow!5},
    framextopmargin=1pt, % these are needed to put a bit of space above and below code snippet
    framexbottommargin=1pt, 
    frame=tb, 
    framerule=0pt,
}



% for table of contents for supplement only
\usepackage{minitoc}
\renewcommand{\ptctitle}{} % No title "Table of Contents" for the toc
\renewcommand \thepart{} % Make the "Part I" text invisible
\renewcommand \partname{}
\usepackage[toc,page,header]{appendix}

% shortcuts

% preferred epsilon
\newcommand{\eps}[0]{\varepsilon}

% Iverson brackets
\usepackage{bm}
\newcommand{\liver}[0]{\bm{[}}
\newcommand{\river}[0]{\bm{]}}

% --- end nadiah added

\title{Worked example (Example 1)}
\author{Nadiah Kristensen}
\date{\today}

\begin{document}
\maketitle

% --- nadiah added
% for shaded results
\colorlet{shadecolor}{blue!3}
%
\newtheoremstyle{mystyle}%                % Name
  {}%                                     % Space above
  {}%                                     % Space below
  {}%                                     % Body font (often italic)
  {}%                                     % Indent amount
  {\bfseries}%                            % Theorem head font
  {.}%                                    % Punctuation after theorem head
  { }%                                    % Space after theorem head, ' ', or \newline
  {\thmname{#1}\thmnumber{ #2}\thmnote{ (#3)}}% % Theorem head spec (can be left empty, meaning `normal')
\theoremstyle{mystyle}
%
\newtheorem{res2}{Result}[section]
\newenvironment{res}
  {\begin{shaded}\begin{res2}}
  {\end{res2}\end{shaded}}
%
\newtheorem{rmk2}{Remark}[section]
\newenvironment{rmk}
  {\begin{shaded}\begin{rmk2}}
  {\end{rmk2}\end{shaded}}

% for examples 
\theoremstyle{definition}
\newtheorem{examplex}{Example}
\newenvironment{exmi}
{\pushQED{\qed}\renewcommand{\qedsymbol}{$\triangleleft$}\examplex}
{\popQED\endexamplex}
\colorlet{shadecolor}{blue!3}
\newenvironment{exm}
{\begin{shaded}\begin{exmi}}
{\end{exmi}\end{shaded}}

% --- end nadiah added

This document refers to the contents of \ttpath{examples/deterministic/example_1/},
which contains a worked example of solving the introspection dynamics for a 2-player game.

%\part{} % Start the document part
%\parttoc
\tableofcontents

\clearpage
\section{Scenario used for this example}

The scenario modelled is a two-player game with parameter values chosen such:
the players are symmetric,
it is always in the players' self-interest to first-author,
it is not in the players' self-interest to coauthor.
The payoff structure for the coauthorship decision is a Prisoner's Dilemma.

For the introspection dynamics, 
I assume that both players will never first-author against their self-interest 
(first-author strategy "\ttpath{never}"),
and therefore only the dynamics of co-authorship strategies are considered.
The space of possible co-authorship strategies considered is:
\ttpath{pavlov_c}, \ttpath{pavlov_d}, \ttpath{all_c}, and \ttpath{all_d}.

\section{Identify permitted action-state transitions and count implementation errors}
\label{S:identify_permitted_action_state_transitions}

As described in 
\ttpath{docs/deterministic/used_to_build.pdf},
when considering within-game transitions between action states with implementation errors,
certain transitions between the deterministic and ultimate action-states 
$A_d \rightarrow A_u$ are permitted and others are not.
For permitted transitions,
the number of errors required to perform the transition is the input into the 
calculation for the probability of that transition occurring,
which in turn is used to calculate the within-game stationary distribution.

Which transitions are permitted and the number of errors required for the transition 
is independent of the parameters of the model apart from the number of players $n$.
Therefore,
a script was written to pre-identify permitted transitions and calculate their number of implementation errors.

The script \ttpath{scripts/deterministic/calc_transitions_errors.py}
produces a \ttpath{.parquet} file for the two-player game,
\ttpath{results/deterministic/transitions_errors_nbr_players_2.parquet}.
The first few rows of the file are shown below:
\begin{lstlisting}
from_state, to_state, is_valid, nbr_errors, nbr_correct
         0,        0,     True,          0,           2
         1,        0,     True,          1,           1
         2,        0,    False,          0,           2
         3,        0,    False,          1,           1
         4,        0,    False,          0,           2
         5,        0,    False,          1,           1
         6,        0,    False,          0,           2
...
\end{lstlisting}

\clearpage
\section{Calculate payoffs to each player in each action state}

The script \ttpath{examples/deterministic/example_1/calc_actions_payoffs.py} 
was used to calculate the payoffs to each player in each possible action-pair state.

Key steps include specifying the game parameters:
\begin{lstlisting}[language=Python]
# game parameters
n = 2  # number of players
b = 2  # maximum benefit of a paper
c_f = 0.9  # cost of first-authoring
c_c = 0.8  # cost of co-authoring

# topic preferences
preferences = np.array([[0], [2]])
assert len(preferences) == n
\end{lstlisting}

Getting every 2-player action state
(makes use of \ttpath{functions/model_fncs.py} imported as \ttpath{mf}):
\begin{lstlisting}[language=Python]
ID_2_actions = mf.get_ID_2_actions(n)
\end{lstlisting}

Calculating payoffs:
\begin{lstlisting}[language=Python]
topics = mf.topics_fnc(params)
paysV = [mf.calc_pays(params, ID_2_actions[ID], topics) for ID in IDs]
\end{lstlisting}

\section{Caclulate within-game deterministic transitions between action states}

Across the rounds of the iterated game,
the strategy of each player determines their action in the next round,
which is based on the actions taken in the previous round. 
The previous-round actions considered include both their own actions 
as well as those of the the other players.
Therefore,
each possible strategy pairing results in a deterministic dynamic
of transitions between different action states.

The script \ttpath{examples/deterministic/example_1/calc_within_game_deterministic_transitions.py}
calculates the deterministic transitions between action states 
for every pairing of co-authorship strategies.
For each coauthor-strategy pair,
the calculation is a two-step process,
as described in \ttpath{docs/deterministic/used_to_build.pdf}.

First,
decisions are made based on self-interest.
The self-interested actions are those with a positive action gain 
(calculated assuming we are moving from non-authorship $a_{i,j} = 0$
to authorship $a_{i,j} = 1$).
In \ttpath{calc_within_game_deterministic_transitions.py},
the positive action gains are identified with
\begin{lstlisting}[language=Python]
action_gains_positive = mf.calc_action_gains_positive(params, ID_2_actions, gammas, alignments)
\end{lstlisting}

\ttpath{action_gains_positive} is an input into 
the function \ttpath{calc_deterministic_transitions()},
which calculates the deterministic transitions matrix 
\begin{lstlisting}[language=Python]
deterministic_transitions_matrix = mf.calc_deterministic_transitions(
    params, actions_2_ID, action_gains_positive, fsauthor_strats, coauthor_strats
)
\end{lstlisting}

The function \ttpath{calc_deterministic_transitions()} is found in 
\ttpath{functions/model_fncs.py},
where the two steps can be seen.
For both first-author and coauthor decisions,
first one determines if the action is in the player's self-interest.
If it is not,
then the first-author or coauthor strategy is used to determine the next action.

The script
\ttpath{examples/deterministic/example_1/plot_all_within_game_deterministic_transitions.py}
prints graphs of the action transitions for coauthor-strategy pairs.
All graphs can be found in PDF files named
\ttpath{examples/deterministic/example_1/within_game_deterministic_transitions_{strategy_1}_Vs_{strategy_2}.pdf},
and two selected examples are shown in Fig.~\ref{f:within_game_deterministic_transitions}.

\begin{figure}[h]
  \centering
  \begin{tabular}{lll}
    (a) \ttpath{pavlov_d}, \ttpath{pavlov_d} & 
    (b) \ttpath{pavlov_d}, \ttpath{all_d} & 
    (c) \ttpath{pavlov_d}, \ttpath{all_c} \\
    \includegraphics[width=0.3\textwidth]{../../examples/deterministic/example_1/within_game_deterministic_transitions_pavlov_d_Vs_pavlov_d.pdf} &
    \includegraphics[width=0.3\textwidth]{../../examples/deterministic/example_1/within_game_deterministic_transitions_pavlov_d_Vs_all_d.pdf} &
    \includegraphics[width=0.3\textwidth]{../../examples/deterministic/example_1/within_game_deterministic_transitions_pavlov_d_Vs_all_c.pdf} \\
  \end{tabular}
  \caption{
    Deterministic transitions between action states selected pairs of strategies. 
    Nodes represent an action-state $A$, which in this 2-player game is a $2 \times 2$ binary matrix,
    and directed edges indicate how the actions of all players change at each round of the game.
  }
  \label{f:within_game_deterministic_transitions}
\end{figure}

When a player pursuing a Pavlov coauthorship strategy is paired with another Pavlov player,
they will settle into a state of mutually coauthorship
(Fig.~\ref{f:within_game_deterministic_transitions}a),
whereas a pairing with a player who never coauthors results in the Pavlov player switching 
between coauthoring and defecting
(Fig.~\ref{f:within_game_deterministic_transitions}b).
Some within-game dynamics have more than one attractor, 
e.g., the pairing of a Pavlov player with a player who always coauthors
(Fig.~\ref{f:within_game_deterministic_transitions}c).
In situations with more than one attractor,
a stationary distribution cannot be found without an additional assumption, 
such as by introducing a small implementation-error (below).

\section{Calculate within-game stationary distribution of action states (analytic solution)}

The script 
\ttpath{examples/deterministic/example_1/calc_within_game_stationary_distributions.py}
calculates 
the stationary distribution of action states 
using the method described \href{https://nadiah.org/2025/11/18/analytic-stationary}{this blog post}.
For each strategy pairing,
first,
the transition matrix with implementation-error terms $\eps$ is generated
\begin{lstlisting}[language=Python]
# get the transition matrix with error terms included (symbolic entries a fnc of eps)
P, eps = mf.symbolic_transitions_with_errors(pre_2_det, fname_error_count)
\end{lstlisting}
Then the analytic stationary distribution as $\eps \rightarrow 0$ is calculated
\begin{lstlisting}[language=Python]
# find the stationary distribution as eps -> 0
stationary_distn_sp, pwr_level = mf.find_stationary_distribution(P, eps)
\end{lstlisting}

The function \ttpath{symbolic_transitions_with_errors()}
\ttpath{functions/model_fncs.py} uses the symbolic mathematics toolbox SymPy
to construct the transition matrix with $\eps$ terms.
The key input needed is the \ttpath{transition_errors_file_name},
which in this example is the file
\ttpath{results/deterministic/transitions_errors_nbr_players_2.parquet}
described in Sect.~\ref{S:identify_permitted_action_state_transitions}.

The function \ttpath{find_stationary_distribution()} takes the symbolic matrix 
\ttpath{P} and the symbolic variable \ttpath{eps}
to construct the simultaneous equations and solve them.
The details of the solution are found 
\href{https://nadiah.org/2025/11/18/analytic-stationary}{in the blog post}.

The stationary distributions are stored in the parquet file 
\ttpath{examples/deterministic/example_1/within_game_stationary_distribution.parquet}.
The first few rows of the 3rd to 6th columns of the file are shown below:
\begin{lstlisting}
pavlov_c pavlov_c [0 0 0 0 0 0 0 0 0 0 0 0 0 0 0 1] [1 1 1 1 1 1 1 1 1 1 1 1 1 1 1 1]
pavlov_c pavlov_d [0 0 0 0 0 0 0 0 0 0 0 0 0 0 0 1] [1 1 1 1 1 1 1 1 1 1 1 1 1 1 1 1]
pavlov_c all_c	  [0 0 0 0 0 0 0 0 0 0 0 1 0 0 0 2] [1 1 1 1 1 1 1 1 1 1 1 3 1 1 1 3]
\end{lstlisting}
The first two columns shown are the Player 1 and Player 2 coauthorship strategies,
respectively,
and the last two columns are lists of the the numerators and denominators 
of the stationary probabilities. 
List indices correspond to action-states indices.
The analytic stationary distribution always returns rational probabilities,
so the stationary probabilities can be stored without loss of accuracy.

Note that the script
\ttpath{calc_within_game_stationary_distributions.py}
uses this method for every strategy-pair scenario,
including those with a deterministic dynamics with only one attractor.
It is likely that it would be more computationally efficient 
to identify single-attractor cases and use another method to find their attractors. 

\section{Calculate longterm expected payoffs}

For each strategy pairing,
the expected payoff to each player is calculated by 
summing the products of their payoff in each action state 
by their probability of being in that action state.
The script 
\ttpath{examples/deterministic/example_1/calc_within_game_stationary_payoffs.py}
performs this calculation.

\section{Calculate introspection dynamics stationary distribution}

To determine how players' strategies evolve over time,
an introspection-dynamics model is used.
The introspection dynamics are also modelled as Markov chain,
so the complete model
layers a Markov chain describing the transitions between players'
strategies on top of the Markov chain describing the within-game transitions 
between action states.

Denote the longterm expected payoff to player $i$ when $i$ plays strategy $x$ and 
$j$ plays strategy $y$ as $\pi_i(\{s_{i,x}, s_{j,y} \})$.
Only one player can update their strategy at a time;
therefore, 
only transitions 
$\{s_{i,x}, s_{j,y} \} \rightarrow \{s_{i,z}, s_{j,y} \}$
are permitted.
Then the probability that player $i$ switches their strategy from $x$ to $z$ 
($x \neq z$) is
\begin{equation}
  \mathbb{P}[ \{s_{i,x}, s_{j,y} \} \rightarrow \{s_{i,z}, s_{j,y} \}] 
  = \left(\frac{1}{2} \right) \left(\frac{1}{m_i-1} \right) 
  \rho \left( \{s_{i,x}, s_{j,y} \}, \{s_{i,z}, s_{j,y} \} \right)
  \label{e:prob_transition}
\end{equation}
where $1/2$ is the probability that player $i$ is chosen to update their strategy,
$m_i-1$ is the number of alternative stragies from among which $i$ chose $z$ as the comparison strategy,
and $\rho()$ is the probability $i$ switches from $x$ to $z$ given that $i$ and $z$ were chosen.

$\rho()$ is a function of the difference in long-term expected payoffs between the two strategies
\begin{equation*}
  \rho \left( \{s_{i,x}, s_{j,y} \}, \{s_{i,z}, s_{j,y} \} \right)
  = \frac{1}{
    1 + \exp(
      -\delta \left[
        \pi_i(\{s_{i,z}, s_{j,y} \}) - \pi_i(\{s_{i,x}, s_{j,y} \})
      \right]
    )
  }
  \label{e:rho}
\end{equation*}
where $\delta$ is the introspection strength.

The transition matrix describing the introspection dynamics is populated with 
$\mathbb{P}[ \{s_{i,x}, s_{j,y} \} \rightarrow \{s_{i,z}, s_{j,y} \}]$.

\subsection{As introspection strength approaches infinity}

The script \ttpath{scripts/deterministic/plot_limit_introspection_stationary_distribution.py}
plots graphs of the introspection dynamics in the limit $\delta \rightarrow \infty$.
As $\delta \rightarrow \infty$,
the function $\rho()$ simplifies to
\begin{equation*}
  \rho \left( \{s_{i,x}, s_{j,y} \}, \{s_{i,z}, s_{j,y} \} \right) =
  \begin{cases}
    0 & \text{if } \pi_i(\{s_{i,z}, s_{j,y} \}) < \pi_i(\{s_{i,x}, s_{j,y} \}) \\
    \frac{1}{2} & \text{if } \pi_i(\{s_{i,z}, s_{j,y} \}) = \pi_i(\{s_{i,x}, s_{j,y} \}) \\
    1 & \text{if } \pi_i(\{s_{i,z}, s_{j,y} \}) > \pi_i(\{s_{i,x}, s_{j,y} \}) 
  \end{cases}
\end{equation*}

The dynamics has two attractors (Fig.~\ref{f:limit_introspection_stationary_distribution}):
a strategy pair playing \ttpath{all_d}, \ttpath{all_d};
and drift between combinations of \ttpath{pavlov_c} and \ttpath{pavlov_d}.

\begin{figure}[h]
  \centering
  \begin{tabular}{ll}
    (a) strategy-pair transition graph & 
    (b) condensation graph of transition graph \\
    \includegraphics[width=0.45\textwidth]{../../examples/deterministic/example_1/limit_introspection_stationary_distribution.pdf} &
    \includegraphics[width=0.45\textwidth]{../../examples/deterministic/example_1/limit_introspection_stationary_distribution_condensation.pdf} \\
  \end{tabular}
  \caption{
    Graphs summarising the introspection dynamics when introspection strength $\delta \rightarrow \infty$.
    (a) Transitions between strategy pairs. 
    Strongly connected components are indicated by nodes with matching colours.
    (b) The condensation graph of the transition graph in panel (a).
  }
  \label{f:limit_introspection_stationary_distribution}
\end{figure}

\subsection{For finite introspection strength}

The script \ttpath{scripts/deterministic/calc_numerical_introspection_stationary_distribution.py}
calculates the stationary distribution of the introspection dynamics for a range of 
finite $\delta$ values.

The first step is to identify the permissible transitions between strategy-pair states.
Then for each permissible transition,
the probability that the transition occurs is calculated using Eq.~\ref{e:prob_transition}
\begin{lstlisting}[language=python]
mat[old_mat_idx, new_mat_idx] = (1 / 2) * (1 / (nbr_strats - 1)) * \
        prob_transition_given_comparison_pair(delta, focal_idx, old_pays, new_pays)
\end{lstlisting}
where the function \ttpath{prob_transition_given_comparison_pair()} implements Eq.~\ref{e:rho}
\begin{lstlisting}[language=python]
def prob_transition_given_comparison_pair(delta, focal_idx, old_pays, new_pays):
    return 1 / (1 + np.exp(-delta * (new_pays[focal_idx] - old_pays[focal_idx])))
\end{lstlisting}

\clearpage
The script
\ttpath{scripts/deterministic/plot_numerical_introspection_stationary_distribution.py}
plots the results from 
\ttpath{calc_numerical_introspection_stationary_distribution.py}
(Fig.~\ref{f:numerical_introspection_stationary_distribution}).
Although (\ttpath{all_d}, \ttpath{all_d}) is an attractor when $\delta \rightarrow \infty$
(Fig.~\ref{f:limit_introspection_stationary_distribution}),
the stationary distribution in the limit appears to be 
drift between combinations of \ttpath{pavlov_c} and \ttpath{pavlov_d}
(Fig.~\ref{f:numerical_introspection_stationary_distribution}).

\begin{figure}[h]
  \begin{minipage}[b]{0.35\textwidth}
    \caption{
      The stationary distribution of strategy-pair probabilities in the introspection dynamics 
      as a function of introspection strength.
    } 
    \label{f:numerical_introspection_stationary_distribution}
  \end{minipage}
  \hfill
  \begin{minipage}[b]{0.60\textwidth}
    \includegraphics[width=\textwidth]{../../examples/deterministic/example_1/numerical_introspection_stationary_distribution.pdf}
  \end{minipage}
\end{figure}


\end{document}
% -------------------------------------------------------------------------------------

\clearpage
\appendix
%
\setcounter{page}{1}
\renewcommand{\thepage}{S\arabic{page}}
\setcounter{secnumdepth}{3}
\setcounter{equation}{0}
\setcounter{section}{0}
\setcounter{table}{0}
\setcounter{figure}{0}
\renewcommand{\thesection}{S\arabic{section}}
\renewcommand{\theequation}{S\arabic{equation}}
\renewcommand{\thefigure}{S\arabic{figure}}
\renewcommand{\thetable}{S\arabic{table}}
%
\addcontentsline{toc}{section}{Appendix} % Add the appendix text to the document TOC
\part{} % Start the appendix part
%
\begin{center} {\bf \Large Supplementary Information} \end{center}
\parttoc
